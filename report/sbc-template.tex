\documentclass[12pt]{article}

\usepackage{sbc-template}


%\usepackage[latin1]{inputenc}  
%\usepackage[utf8]{inputenc}  

\usepackage{graphicx,url}
\usepackage[brazil]{babel}
\usepackage[T1]{fontenc}
\usepackage{amsmath}
\usepackage{listings}
\usepackage{float}

\usepackage{booktabs}
\usepackage{multirow}
\usepackage{siunitx}
\usepackage[export]{adjustbox}

\sloppy

\title{IAR0001 - 2017/1\\Relat�rio Trabalho 4\\Simulated Annealing}

\author{Alexandre Maros\inst{1} }

\address{Departamento de Ci�ncia da Computa��o -- Universidade do Estado de Santa Catarina\\
  Centro de Ci�ncias Tecnol�gicas -- Joinville -- SC -- Brasil
  \email{alehstk@gmail.com}
}

\begin{document} 

\maketitle

%\begin{abstract}
    %Abstract
%\end{abstract}
     
\begin{resumo} 
    a
\end{resumo}

% 1. Introdu��o
%   Contextualiza��o do problema, justificativa, objetivos, estrutura do relat�rio.
\section{Introdu��o}

a

% 2. Problem�tica
%   Detalhamento do problema, PEAS e caracter�sticas do problema
\section{Problem�tica}

b

% 3. Modelo implementado
%   Estrat�gias utilizadas, f�rmulas, defini��es de implementa��o, linguagem
\section{Modelo implementado, experimentos e an�lises}


\begin{table}[h]
    \caption{Resultados obtidos}\label{tab:tab1}
    \centering
    \begin{tabular}{lSSSSSS}
        \toprule
        \multirow{2}{*}{Busca} &
            \multicolumn{2}{c}{Simulated Annealing} &
            \multicolumn{2}{c}{Random Search} \\
            & {$\bar{x}$} & {$\sigma$} & {$\bar{x}$} & {$\sigma$} \\
            \midrule
        uf20-01 & 91 & 0 & 90 & 0 \\
        uf100-01 & 427.9 & 0.567 & 402.7 & 1.828 \\
        uf250-01 & 1036.5 & 3.027 & 978.5 & 1.433 \\
        \bottomrule
  \end{tabular}
\end{table}


\begin{table}[h]
    \caption{Resultados obtidos}\label{tab:tab1}
    \centering
    \begin{tabular}{lSSSSSS}
        \toprule
        \multirow{2}{*}{Busca} &
            \multicolumn{2}{c}{Simulated Annealing} &
            \multicolumn{2}{c}{Random Search} \\
            & {$\bar{x}$} & {$\sigma$} & {$\bar{x}$} & {$\sigma$} \\
            \midrule
        uf20-01 & 91 & 0 & 90 & 0 \\
        uf100-01 & 430 & 0 & 402.7 & 1.828 \\
        uf250-01 & 1064.3 & 1.059 & 978.5 & 1.433 \\
        \bottomrule
  \end{tabular}
\end{table}

%\begin{figure}[h]
    %\centering
    %\includegraphics[width=1\textwidth]{figuras/plot1}
    %\caption{Gr�ficos dos Testes}
    %\label{fig:plot1}
%\end{figure}


% 5. Conclus�o
%   Considera��es sobre o trabalho e sobre os resultados obtidos, trabalhos futuros.
\section{Conclus�o}

conclu

\bibliographystyle{sbc}
\bibliography{sbc-template}

\end{document}
